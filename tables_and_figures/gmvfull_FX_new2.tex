% latex table generated in R 4.3.2 by xtable 1.8-4 package
% Fri Jun 28 11:43:14 2024
\begin{table}[ht]
\centering
\scalebox{0.8}{
\begin{tabular}{lccc|ccc}
  \hline
 & Jore1 & Geweke & Equal & CAW & DCC-t & DCC-HEAVY-t \\ 
  \hline
avrg & 6.766 & 6.734 & 6.790 & 6.770 & 6.752 & 6.687 \\ 
  sd & 1.334 & 1.356 & 1.357 & 1.345 & 1.368 & 1.345 \\ 
  X.p.value. & 1.000 & 0.265 & 0.111 & 0.560 & 0.178 & 0.905 \\ 
  SR & 3.708 & 3.627 & 3.665 & 3.682 & 3.606 & 3.621 \\ 
  rCVaR05 & -40.488 & -40.758 & -40.508 & -40.760 & -41.947 & -41.194 \\ 
  rCVaR10 & -31.260 & -31.422 & -31.240 & -31.504 & -31.763 & -31.527 \\ 
  TO & 0.466 & 0.465 & 0.466 & 0.463 & 0.470 & 0.469 \\ 
  CO & 0.419 & 0.417 & 0.420 & 0.409 & 0.416 & 0.411 \\ 
  avrg.1 & 6.856 & 6.782 & 6.867 & 6.835 & 6.781 & 6.718 \\ 
  sd.1 & 1.337 & 1.361 & 1.364 & 1.351 & 1.373 & 1.351 \\ 
  X.p.value..1 & 1.000 & 0.199 & 0.043 & 0.381 & 0.133 & 0.660 \\ 
  SR.1 & 3.768 & 3.648 & 3.701 & 3.713 & 3.615 & 3.628 \\ 
  rCVaR05.1 & -40.430 & -40.878 & -40.644 & -40.882 & -42.110 & -41.494 \\ 
  rCVaR10.1 & -31.061 & -31.385 & -31.314 & -31.458 & -31.812 & -31.667 \\ 
  TO.1 & 0.467 & 0.465 & 0.467 & 0.463 & 0.471 & 0.470 \\ 
  CO.1 & 0.419 & 0.418 & 0.420 & 0.410 & 0.417 & 0.412 \\ 
  avrg.2 & 6.839 & 6.787 & 6.850 & 6.827 & 6.773 & 6.730 \\ 
  sd.2 & 1.340 & 1.359 & 1.365 & 1.349 & 1.374 & 1.353 \\ 
  X.p.value..2 & 1.000 & 0.322 & 0.040 & 0.552 & 0.154 & 0.654 \\ 
  SR.2 & 3.746 & 3.657 & 3.686 & 3.712 & 3.607 & 3.631 \\ 
  rCVaR05.2 & -40.641 & -40.985 & -40.847 & -40.953 & -42.219 & -41.567 \\ 
  rCVaR10.2 & -31.230 & -31.412 & -31.421 & -31.541 & -31.879 & -31.675 \\ 
  TO.2 & 0.467 & 0.465 & 0.467 & 0.463 & 0.471 & 0.470 \\ 
  CO.2 & 0.419 & 0.417 & 0.420 & 0.410 & 0.417 & 0.412 \\ 
   \hline
\end{tabular}
}
\caption{Portfolio allocation results based on 1-step-ahead
              predictions for 2020/01/02 to 2023/01/31 out-of-sample period ($K$ = 797 observations).
              The three portfolios are:
             Global Minimum Variance (GMV), and minimum Conditional Value at Risk
             for lower 5 and 10 percentiles (CVaR05 and CVaR10), all with short-sale constraints.
             The table reports average portfolio return (avrg),
             overall standard deviation in \% (sd),
             the p-value corresponding to the model confidence set of
             Hansen, Lunde and Nason (2011) using a significance level of 5\%,
             adjusted Sharpe ratio (SR), realized Conditional Value at Risk
             for lower 5 and 10 percentiles  (rCVaR05 and rCVaR10),
             turnover (TO), and concentration (CO), all quantities annualized, for the
             pooled models (Geweke's, Jore's and
             equally weighted),
            two best individual models (CAW  and DCC-t) and a competitor model (DCC-HEAVY-t).
            In gray is highlighted the best performing portfolio and the second best is underlined.} 
\label{table:gmvfull_FX_new2}
\end{table}

